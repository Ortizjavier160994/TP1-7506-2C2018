
% Default to the notebook output style

    


% Inherit from the specified cell style.




    
\documentclass[11pt]{article}

    
    
    \usepackage[T1]{fontenc}
    % Nicer default font (+ math font) than Computer Modern for most use cases
    \usepackage{mathpazo}

    % Basic figure setup, for now with no caption control since it's done
    % automatically by Pandoc (which extracts ![](path) syntax from Markdown).
    \usepackage{graphicx}
    % We will generate all images so they have a width \maxwidth. This means
    % that they will get their normal width if they fit onto the page, but
    % are scaled down if they would overflow the margins.
    \makeatletter
    \def\maxwidth{\ifdim\Gin@nat@width>\linewidth\linewidth
    \else\Gin@nat@width\fi}
    \makeatother
    \let\Oldincludegraphics\includegraphics
    % Set max figure width to be 80% of text width, for now hardcoded.
    \renewcommand{\includegraphics}[1]{\Oldincludegraphics[width=.8\maxwidth]{#1}}
    % Ensure that by default, figures have no caption (until we provide a
    % proper Figure object with a Caption API and a way to capture that
    % in the conversion process - todo).
    \usepackage{caption}
    \DeclareCaptionLabelFormat{nolabel}{}
    \captionsetup{labelformat=nolabel}

    \usepackage{adjustbox} % Used to constrain images to a maximum size 
    \usepackage{xcolor} % Allow colors to be defined
    \usepackage{enumerate} % Needed for markdown enumerations to work
    \usepackage{geometry} % Used to adjust the document margins
    \usepackage{amsmath} % Equations
    \usepackage{amssymb} % Equations
    \usepackage{textcomp} % defines textquotesingle
    % Hack from http://tex.stackexchange.com/a/47451/13684:
    \AtBeginDocument{%
        \def\PYZsq{\textquotesingle}% Upright quotes in Pygmentized code
    }
    \usepackage{upquote} % Upright quotes for verbatim code
    \usepackage{eurosym} % defines \euro
    \usepackage[mathletters]{ucs} % Extended unicode (utf-8) support
    \usepackage[utf8x]{inputenc} % Allow utf-8 characters in the tex document
    \usepackage{fancyvrb} % verbatim replacement that allows latex
    \usepackage{grffile} % extends the file name processing of package graphics 
                         % to support a larger range 
    % The hyperref package gives us a pdf with properly built
    % internal navigation ('pdf bookmarks' for the table of contents,
    % internal cross-reference links, web links for URLs, etc.)
    \usepackage{hyperref}
    \usepackage{longtable} % longtable support required by pandoc >1.10
    \usepackage{booktabs}  % table support for pandoc > 1.12.2
    \usepackage[inline]{enumitem} % IRkernel/repr support (it uses the enumerate* environment)
    \usepackage[normalem]{ulem} % ulem is needed to support strikethroughs (\sout)
                                % normalem makes italics be italics, not underlines
    

    
    
    % Colors for the hyperref package
    \definecolor{urlcolor}{rgb}{0,.145,.698}
    \definecolor{linkcolor}{rgb}{.71,0.21,0.01}
    \definecolor{citecolor}{rgb}{.12,.54,.11}

    % ANSI colors
    \definecolor{ansi-black}{HTML}{3E424D}
    \definecolor{ansi-black-intense}{HTML}{282C36}
    \definecolor{ansi-red}{HTML}{E75C58}
    \definecolor{ansi-red-intense}{HTML}{B22B31}
    \definecolor{ansi-green}{HTML}{00A250}
    \definecolor{ansi-green-intense}{HTML}{007427}
    \definecolor{ansi-yellow}{HTML}{DDB62B}
    \definecolor{ansi-yellow-intense}{HTML}{B27D12}
    \definecolor{ansi-blue}{HTML}{208FFB}
    \definecolor{ansi-blue-intense}{HTML}{0065CA}
    \definecolor{ansi-magenta}{HTML}{D160C4}
    \definecolor{ansi-magenta-intense}{HTML}{A03196}
    \definecolor{ansi-cyan}{HTML}{60C6C8}
    \definecolor{ansi-cyan-intense}{HTML}{258F8F}
    \definecolor{ansi-white}{HTML}{C5C1B4}
    \definecolor{ansi-white-intense}{HTML}{A1A6B2}

    % commands and environments needed by pandoc snippets
    % extracted from the output of `pandoc -s`
    \providecommand{\tightlist}{%
      \setlength{\itemsep}{0pt}\setlength{\parskip}{0pt}}
    \DefineVerbatimEnvironment{Highlighting}{Verbatim}{commandchars=\\\{\}}
    % Add ',fontsize=\small' for more characters per line
    \newenvironment{Shaded}{}{}
    \newcommand{\KeywordTok}[1]{\textcolor[rgb]{0.00,0.44,0.13}{\textbf{{#1}}}}
    \newcommand{\DataTypeTok}[1]{\textcolor[rgb]{0.56,0.13,0.00}{{#1}}}
    \newcommand{\DecValTok}[1]{\textcolor[rgb]{0.25,0.63,0.44}{{#1}}}
    \newcommand{\BaseNTok}[1]{\textcolor[rgb]{0.25,0.63,0.44}{{#1}}}
    \newcommand{\FloatTok}[1]{\textcolor[rgb]{0.25,0.63,0.44}{{#1}}}
    \newcommand{\CharTok}[1]{\textcolor[rgb]{0.25,0.44,0.63}{{#1}}}
    \newcommand{\StringTok}[1]{\textcolor[rgb]{0.25,0.44,0.63}{{#1}}}
    \newcommand{\CommentTok}[1]{\textcolor[rgb]{0.38,0.63,0.69}{\textit{{#1}}}}
    \newcommand{\OtherTok}[1]{\textcolor[rgb]{0.00,0.44,0.13}{{#1}}}
    \newcommand{\AlertTok}[1]{\textcolor[rgb]{1.00,0.00,0.00}{\textbf{{#1}}}}
    \newcommand{\FunctionTok}[1]{\textcolor[rgb]{0.02,0.16,0.49}{{#1}}}
    \newcommand{\RegionMarkerTok}[1]{{#1}}
    \newcommand{\ErrorTok}[1]{\textcolor[rgb]{1.00,0.00,0.00}{\textbf{{#1}}}}
    \newcommand{\NormalTok}[1]{{#1}}
    
    % Additional commands for more recent versions of Pandoc
    \newcommand{\ConstantTok}[1]{\textcolor[rgb]{0.53,0.00,0.00}{{#1}}}
    \newcommand{\SpecialCharTok}[1]{\textcolor[rgb]{0.25,0.44,0.63}{{#1}}}
    \newcommand{\VerbatimStringTok}[1]{\textcolor[rgb]{0.25,0.44,0.63}{{#1}}}
    \newcommand{\SpecialStringTok}[1]{\textcolor[rgb]{0.73,0.40,0.53}{{#1}}}
    \newcommand{\ImportTok}[1]{{#1}}
    \newcommand{\DocumentationTok}[1]{\textcolor[rgb]{0.73,0.13,0.13}{\textit{{#1}}}}
    \newcommand{\AnnotationTok}[1]{\textcolor[rgb]{0.38,0.63,0.69}{\textbf{\textit{{#1}}}}}
    \newcommand{\CommentVarTok}[1]{\textcolor[rgb]{0.38,0.63,0.69}{\textbf{\textit{{#1}}}}}
    \newcommand{\VariableTok}[1]{\textcolor[rgb]{0.10,0.09,0.49}{{#1}}}
    \newcommand{\ControlFlowTok}[1]{\textcolor[rgb]{0.00,0.44,0.13}{\textbf{{#1}}}}
    \newcommand{\OperatorTok}[1]{\textcolor[rgb]{0.40,0.40,0.40}{{#1}}}
    \newcommand{\BuiltInTok}[1]{{#1}}
    \newcommand{\ExtensionTok}[1]{{#1}}
    \newcommand{\PreprocessorTok}[1]{\textcolor[rgb]{0.74,0.48,0.00}{{#1}}}
    \newcommand{\AttributeTok}[1]{\textcolor[rgb]{0.49,0.56,0.16}{{#1}}}
    \newcommand{\InformationTok}[1]{\textcolor[rgb]{0.38,0.63,0.69}{\textbf{\textit{{#1}}}}}
    \newcommand{\WarningTok}[1]{\textcolor[rgb]{0.38,0.63,0.69}{\textbf{\textit{{#1}}}}}
    
    
    % Define a nice break command that doesn't care if a line doesn't already
    % exist.
    \def\br{\hspace*{\fill} \\* }
    % Math Jax compatability definitions
    \def\gt{>}
    \def\lt{<}
    % Document parameters
    \title{TP1}
    
    
    

    % Pygments definitions
    
\makeatletter
\def\PY@reset{\let\PY@it=\relax \let\PY@bf=\relax%
    \let\PY@ul=\relax \let\PY@tc=\relax%
    \let\PY@bc=\relax \let\PY@ff=\relax}
\def\PY@tok#1{\csname PY@tok@#1\endcsname}
\def\PY@toks#1+{\ifx\relax#1\empty\else%
    \PY@tok{#1}\expandafter\PY@toks\fi}
\def\PY@do#1{\PY@bc{\PY@tc{\PY@ul{%
    \PY@it{\PY@bf{\PY@ff{#1}}}}}}}
\def\PY#1#2{\PY@reset\PY@toks#1+\relax+\PY@do{#2}}

\expandafter\def\csname PY@tok@w\endcsname{\def\PY@tc##1{\textcolor[rgb]{0.73,0.73,0.73}{##1}}}
\expandafter\def\csname PY@tok@c\endcsname{\let\PY@it=\textit\def\PY@tc##1{\textcolor[rgb]{0.25,0.50,0.50}{##1}}}
\expandafter\def\csname PY@tok@cp\endcsname{\def\PY@tc##1{\textcolor[rgb]{0.74,0.48,0.00}{##1}}}
\expandafter\def\csname PY@tok@k\endcsname{\let\PY@bf=\textbf\def\PY@tc##1{\textcolor[rgb]{0.00,0.50,0.00}{##1}}}
\expandafter\def\csname PY@tok@kp\endcsname{\def\PY@tc##1{\textcolor[rgb]{0.00,0.50,0.00}{##1}}}
\expandafter\def\csname PY@tok@kt\endcsname{\def\PY@tc##1{\textcolor[rgb]{0.69,0.00,0.25}{##1}}}
\expandafter\def\csname PY@tok@o\endcsname{\def\PY@tc##1{\textcolor[rgb]{0.40,0.40,0.40}{##1}}}
\expandafter\def\csname PY@tok@ow\endcsname{\let\PY@bf=\textbf\def\PY@tc##1{\textcolor[rgb]{0.67,0.13,1.00}{##1}}}
\expandafter\def\csname PY@tok@nb\endcsname{\def\PY@tc##1{\textcolor[rgb]{0.00,0.50,0.00}{##1}}}
\expandafter\def\csname PY@tok@nf\endcsname{\def\PY@tc##1{\textcolor[rgb]{0.00,0.00,1.00}{##1}}}
\expandafter\def\csname PY@tok@nc\endcsname{\let\PY@bf=\textbf\def\PY@tc##1{\textcolor[rgb]{0.00,0.00,1.00}{##1}}}
\expandafter\def\csname PY@tok@nn\endcsname{\let\PY@bf=\textbf\def\PY@tc##1{\textcolor[rgb]{0.00,0.00,1.00}{##1}}}
\expandafter\def\csname PY@tok@ne\endcsname{\let\PY@bf=\textbf\def\PY@tc##1{\textcolor[rgb]{0.82,0.25,0.23}{##1}}}
\expandafter\def\csname PY@tok@nv\endcsname{\def\PY@tc##1{\textcolor[rgb]{0.10,0.09,0.49}{##1}}}
\expandafter\def\csname PY@tok@no\endcsname{\def\PY@tc##1{\textcolor[rgb]{0.53,0.00,0.00}{##1}}}
\expandafter\def\csname PY@tok@nl\endcsname{\def\PY@tc##1{\textcolor[rgb]{0.63,0.63,0.00}{##1}}}
\expandafter\def\csname PY@tok@ni\endcsname{\let\PY@bf=\textbf\def\PY@tc##1{\textcolor[rgb]{0.60,0.60,0.60}{##1}}}
\expandafter\def\csname PY@tok@na\endcsname{\def\PY@tc##1{\textcolor[rgb]{0.49,0.56,0.16}{##1}}}
\expandafter\def\csname PY@tok@nt\endcsname{\let\PY@bf=\textbf\def\PY@tc##1{\textcolor[rgb]{0.00,0.50,0.00}{##1}}}
\expandafter\def\csname PY@tok@nd\endcsname{\def\PY@tc##1{\textcolor[rgb]{0.67,0.13,1.00}{##1}}}
\expandafter\def\csname PY@tok@s\endcsname{\def\PY@tc##1{\textcolor[rgb]{0.73,0.13,0.13}{##1}}}
\expandafter\def\csname PY@tok@sd\endcsname{\let\PY@it=\textit\def\PY@tc##1{\textcolor[rgb]{0.73,0.13,0.13}{##1}}}
\expandafter\def\csname PY@tok@si\endcsname{\let\PY@bf=\textbf\def\PY@tc##1{\textcolor[rgb]{0.73,0.40,0.53}{##1}}}
\expandafter\def\csname PY@tok@se\endcsname{\let\PY@bf=\textbf\def\PY@tc##1{\textcolor[rgb]{0.73,0.40,0.13}{##1}}}
\expandafter\def\csname PY@tok@sr\endcsname{\def\PY@tc##1{\textcolor[rgb]{0.73,0.40,0.53}{##1}}}
\expandafter\def\csname PY@tok@ss\endcsname{\def\PY@tc##1{\textcolor[rgb]{0.10,0.09,0.49}{##1}}}
\expandafter\def\csname PY@tok@sx\endcsname{\def\PY@tc##1{\textcolor[rgb]{0.00,0.50,0.00}{##1}}}
\expandafter\def\csname PY@tok@m\endcsname{\def\PY@tc##1{\textcolor[rgb]{0.40,0.40,0.40}{##1}}}
\expandafter\def\csname PY@tok@gh\endcsname{\let\PY@bf=\textbf\def\PY@tc##1{\textcolor[rgb]{0.00,0.00,0.50}{##1}}}
\expandafter\def\csname PY@tok@gu\endcsname{\let\PY@bf=\textbf\def\PY@tc##1{\textcolor[rgb]{0.50,0.00,0.50}{##1}}}
\expandafter\def\csname PY@tok@gd\endcsname{\def\PY@tc##1{\textcolor[rgb]{0.63,0.00,0.00}{##1}}}
\expandafter\def\csname PY@tok@gi\endcsname{\def\PY@tc##1{\textcolor[rgb]{0.00,0.63,0.00}{##1}}}
\expandafter\def\csname PY@tok@gr\endcsname{\def\PY@tc##1{\textcolor[rgb]{1.00,0.00,0.00}{##1}}}
\expandafter\def\csname PY@tok@ge\endcsname{\let\PY@it=\textit}
\expandafter\def\csname PY@tok@gs\endcsname{\let\PY@bf=\textbf}
\expandafter\def\csname PY@tok@gp\endcsname{\let\PY@bf=\textbf\def\PY@tc##1{\textcolor[rgb]{0.00,0.00,0.50}{##1}}}
\expandafter\def\csname PY@tok@go\endcsname{\def\PY@tc##1{\textcolor[rgb]{0.53,0.53,0.53}{##1}}}
\expandafter\def\csname PY@tok@gt\endcsname{\def\PY@tc##1{\textcolor[rgb]{0.00,0.27,0.87}{##1}}}
\expandafter\def\csname PY@tok@err\endcsname{\def\PY@bc##1{\setlength{\fboxsep}{0pt}\fcolorbox[rgb]{1.00,0.00,0.00}{1,1,1}{\strut ##1}}}
\expandafter\def\csname PY@tok@kc\endcsname{\let\PY@bf=\textbf\def\PY@tc##1{\textcolor[rgb]{0.00,0.50,0.00}{##1}}}
\expandafter\def\csname PY@tok@kd\endcsname{\let\PY@bf=\textbf\def\PY@tc##1{\textcolor[rgb]{0.00,0.50,0.00}{##1}}}
\expandafter\def\csname PY@tok@kn\endcsname{\let\PY@bf=\textbf\def\PY@tc##1{\textcolor[rgb]{0.00,0.50,0.00}{##1}}}
\expandafter\def\csname PY@tok@kr\endcsname{\let\PY@bf=\textbf\def\PY@tc##1{\textcolor[rgb]{0.00,0.50,0.00}{##1}}}
\expandafter\def\csname PY@tok@bp\endcsname{\def\PY@tc##1{\textcolor[rgb]{0.00,0.50,0.00}{##1}}}
\expandafter\def\csname PY@tok@fm\endcsname{\def\PY@tc##1{\textcolor[rgb]{0.00,0.00,1.00}{##1}}}
\expandafter\def\csname PY@tok@vc\endcsname{\def\PY@tc##1{\textcolor[rgb]{0.10,0.09,0.49}{##1}}}
\expandafter\def\csname PY@tok@vg\endcsname{\def\PY@tc##1{\textcolor[rgb]{0.10,0.09,0.49}{##1}}}
\expandafter\def\csname PY@tok@vi\endcsname{\def\PY@tc##1{\textcolor[rgb]{0.10,0.09,0.49}{##1}}}
\expandafter\def\csname PY@tok@vm\endcsname{\def\PY@tc##1{\textcolor[rgb]{0.10,0.09,0.49}{##1}}}
\expandafter\def\csname PY@tok@sa\endcsname{\def\PY@tc##1{\textcolor[rgb]{0.73,0.13,0.13}{##1}}}
\expandafter\def\csname PY@tok@sb\endcsname{\def\PY@tc##1{\textcolor[rgb]{0.73,0.13,0.13}{##1}}}
\expandafter\def\csname PY@tok@sc\endcsname{\def\PY@tc##1{\textcolor[rgb]{0.73,0.13,0.13}{##1}}}
\expandafter\def\csname PY@tok@dl\endcsname{\def\PY@tc##1{\textcolor[rgb]{0.73,0.13,0.13}{##1}}}
\expandafter\def\csname PY@tok@s2\endcsname{\def\PY@tc##1{\textcolor[rgb]{0.73,0.13,0.13}{##1}}}
\expandafter\def\csname PY@tok@sh\endcsname{\def\PY@tc##1{\textcolor[rgb]{0.73,0.13,0.13}{##1}}}
\expandafter\def\csname PY@tok@s1\endcsname{\def\PY@tc##1{\textcolor[rgb]{0.73,0.13,0.13}{##1}}}
\expandafter\def\csname PY@tok@mb\endcsname{\def\PY@tc##1{\textcolor[rgb]{0.40,0.40,0.40}{##1}}}
\expandafter\def\csname PY@tok@mf\endcsname{\def\PY@tc##1{\textcolor[rgb]{0.40,0.40,0.40}{##1}}}
\expandafter\def\csname PY@tok@mh\endcsname{\def\PY@tc##1{\textcolor[rgb]{0.40,0.40,0.40}{##1}}}
\expandafter\def\csname PY@tok@mi\endcsname{\def\PY@tc##1{\textcolor[rgb]{0.40,0.40,0.40}{##1}}}
\expandafter\def\csname PY@tok@il\endcsname{\def\PY@tc##1{\textcolor[rgb]{0.40,0.40,0.40}{##1}}}
\expandafter\def\csname PY@tok@mo\endcsname{\def\PY@tc##1{\textcolor[rgb]{0.40,0.40,0.40}{##1}}}
\expandafter\def\csname PY@tok@ch\endcsname{\let\PY@it=\textit\def\PY@tc##1{\textcolor[rgb]{0.25,0.50,0.50}{##1}}}
\expandafter\def\csname PY@tok@cm\endcsname{\let\PY@it=\textit\def\PY@tc##1{\textcolor[rgb]{0.25,0.50,0.50}{##1}}}
\expandafter\def\csname PY@tok@cpf\endcsname{\let\PY@it=\textit\def\PY@tc##1{\textcolor[rgb]{0.25,0.50,0.50}{##1}}}
\expandafter\def\csname PY@tok@c1\endcsname{\let\PY@it=\textit\def\PY@tc##1{\textcolor[rgb]{0.25,0.50,0.50}{##1}}}
\expandafter\def\csname PY@tok@cs\endcsname{\let\PY@it=\textit\def\PY@tc##1{\textcolor[rgb]{0.25,0.50,0.50}{##1}}}

\def\PYZbs{\char`\\}
\def\PYZus{\char`\_}
\def\PYZob{\char`\{}
\def\PYZcb{\char`\}}
\def\PYZca{\char`\^}
\def\PYZam{\char`\&}
\def\PYZlt{\char`\<}
\def\PYZgt{\char`\>}
\def\PYZsh{\char`\#}
\def\PYZpc{\char`\%}
\def\PYZdl{\char`\$}
\def\PYZhy{\char`\-}
\def\PYZsq{\char`\'}
\def\PYZdq{\char`\"}
\def\PYZti{\char`\~}
% for compatibility with earlier versions
\def\PYZat{@}
\def\PYZlb{[}
\def\PYZrb{]}
\makeatother


    % Exact colors from NB
    \definecolor{incolor}{rgb}{0.0, 0.0, 0.5}
    \definecolor{outcolor}{rgb}{0.545, 0.0, 0.0}



    
    % Prevent overflowing lines due to hard-to-break entities
    \sloppy 
    % Setup hyperref package
    \hypersetup{
      breaklinks=true,  % so long urls are correctly broken across lines
      colorlinks=true,
      urlcolor=urlcolor,
      linkcolor=linkcolor,
      citecolor=citecolor,
      }
    % Slightly bigger margins than the latex defaults
    
    \geometry{verbose,tmargin=1in,bmargin=1in,lmargin=1in,rmargin=1in}
    
    

    \begin{document}
    
    
    \maketitle
    
\hypertarget{anuxe1lisis-exploratorio}{%
\section{Análisis exploratorio}\label{anuxe1lisis-exploratorio}}

    \hypertarget{overview-del-set-de-datos}{%
\subsection{Overview del set de datos}\label{overview-del-set-de-datos}}


    \begin{Verbatim}[commandchars=\\\{\}]
\PY{n+nb}{print}\PY{p}{(}\PY{n}{datos}\PY{o}{.}\PY{n}{shape}\PY{p}{)}
\end{Verbatim}


    \begin{Verbatim}[commandchars=\\\{\}]
(1011288, 23)

    \end{Verbatim}

\begin{Verbatim}[commandchars=\\\{\}]
\PY{n}{datos}\PY{o}{.}\PY{n}{isnull}\PY{p}{(}\PY{p}{)}\PY{o}{.}\PY{n}{sum}\PY{p}{(}\PY{p}{)}
        timestamp                         0
        event                             0
        person                            0
        url                          928532
        sku                          447452
        model                        447004
        condition                    447452
        storage                      447452
        color                        447452
        skus                         789589
        search\_term                  962321
        staticpage                  1007690
        campaign\_source              928492
        search\_engine                960331
        channel                      923910
        new\_vs\_returning             923910
        city                         923910
        region                       923910
        country                      923910
        device\_type                  923910
        screen\_resolution            923910
        operating\_system\_version     923910
        browser\_version              923910
        dtype: int64
\end{Verbatim}
            
    \hypertarget{muchos-nulos.-staticpage-tiene-casi-todos-nulos.-no-aporta-nada-de-informaciuxf3n}{%
\subparagraph{Muchos nulos. `staticpage' tiene casi todos nulos. No
aporta nada de
información}\label{muchos-nulos.-staticpage-tiene-casi-todos-nulos.-no-aporta-nada-de-informaciuxf3n}}

    \hypertarget{eventos}{%
\paragraph{Eventos}\label{eventos}}

    \begin{Verbatim}[commandchars=\\\{\}]
 \PY{k}{for} \PY{n}{x} \PY{o+ow}{in} \PY{n}{datos}\PY{p}{[}\PY{l+s+s1}{\PYZsq{}}\PY{l+s+s1}{event}\PY{l+s+s1}{\PYZsq{}}\PY{p}{]}\PY{o}{.}\PY{n}{unique}\PY{p}{(}\PY{p}{)}\PY{p}{:}
    \PY{n+nb}{print} \PY{p}{(}\PY{n}{x}\PY{p}{)}
\end{Verbatim}


    \begin{Verbatim}[commandchars=\\\{\}]
ad campaign hit
visited site
viewed product
checkout
generic listing
search engine hit
brand listing
searched products
conversion
staticpage
lead

    \end{Verbatim}

    \hypertarget{condiciones-en-las-que-puede-encontrarse-un-producto}{%
\paragraph{Condiciones en las que puede encontrarse un
producto}\label{condiciones-en-las-que-puede-encontrarse-un-producto}}

    \begin{Verbatim}[commandchars=\\\{\}]
\PY{k}{for} \PY{n}{x} \PY{o+ow}{in} \PY{n}{datos}\PY{p}{[}\PY{l+s+s1}{\PYZsq{}}\PY{l+s+s1}{condition}\PY{l+s+s1}{\PYZsq{}}\PY{p}{]}\PY{o}{.}\PY{n}{unique}\PY{p}{(}\PY{p}{)}\PY{p}{:}
    \PY{n+nb}{print} \PY{p}{(}\PY{n}{x}\PY{p}{)}
\end{Verbatim}


    \begin{Verbatim}[commandchars=\\\{\}]
    \PY{c+c1}{\PYZsh{} Las pasamos a español para que las visualizaciones se vean en español}
nan
Bom \PY{o}{=}\PY{l+s+s1}{\PYZsq{}}\PY{l+s+s1}{Bien}\PY{l+s+s1}{\PYZsq{}}
Muito Bom \PY{o}{=}\PY{l+s+s1}{\PYZsq{}}\PY{l+s+s1}{Muy Bien}\PY{l+s+s1}{\PYZsq{}}
Excelente
Bom - Sem Touch ID \PY{o}{=}\PY{l+s+s1}{\PYZsq{}}\PY{l+s+s1}{Bien - Sin Touch}\PY{l+s+s1}{\PYZsq{}}
Novo \PY{o}{=}\PY{l+s+s1}{\PYZsq{}}\PY{l+s+s1}{Nuevo}\PY{l+s+s1}{\PYZsq{}}

    \end{Verbatim}

    \hypertarget{visualizaciones}{%
\subsection{Visualizaciones}\label{visualizaciones}}

    \hypertarget{las-empresas-publicitarias-que-muxe1s-vistas-generan}{%
\subsubsection{1. Las empresas publicitarias que más vistas
generan}\label{las-empresas-publicitarias-que-muxe1s-vistas-generan}}

    \begin{center}
    \adjustimage{max size={0.9\linewidth}{0.9\paperheight}}{output_21_1.png}
    \end{center}
    \hypertarget{modelos-muxe1s-visitados}{%
\subsubsection{2. Modelos más
visitados}\label{modelos-muxe1s-visitados}}

    \begin{center}
    \adjustimage{max size={0.9\linewidth}{0.9\paperheight}}{output_23_1.png}
    \end{center}

    \hypertarget{los-teluxe9fonos-muxe1s-buscados-son-los-iphone-y-los-samsung}{%
\subparagraph{Los teléfonos más buscados son los iphone y los
samsung}\label{los-teluxe9fonos-muxe1s-buscados-son-los-iphone-y-los-samsung}}

    \hypertarget{modelos-muxe1s-comprados}{%
\subsubsection{3. Modelos más
comprados}\label{modelos-muxe1s-comprados}}

    \begin{center}
    \adjustimage{max size={0.9\linewidth}{0.9\paperheight}}{output_26_1.png}
    \end{center}
    
    \hypertarget{los-teluxe9fonos-muxe1s-comprados-son-los-iphone-seguidos-por-los-samsung}{%
\subparagraph{Los teléfonos más comprados son los iphone, seguidos por
los
samsung}\label{los-teluxe9fonos-muxe1s-comprados-son-los-iphone-seguidos-por-los-samsung}}

    \hypertarget{estado-de-los-productos-muxe1s-comprados}{%
\subsubsection{4. Estado de los productos más
comprados}\label{estado-de-los-productos-muxe1s-comprados}}

    \begin{center}
    \adjustimage{max size={0.9\linewidth}{0.9\paperheight}}{output_29_1.png}
    \end{center}

    \hypertarget{la-gente-no-le-da-tanta-importancia-al-estado-del-teluxe9fono}{%
\subparagraph{La gente no le da tanta importancia al estado del
teléfono}\label{la-gente-no-le-da-tanta-importancia-al-estado-del-teluxe9fono}}

    \hypertarget{estado-de-los-productos-muxe1s-visitiados}{%
\subsubsection{5. Estado de los productos más
visitiados}\label{estado-de-los-productos-muxe1s-visitiados}}

    
    \begin{center}
    \adjustimage{max size={0.9\linewidth}{0.9\paperheight}}{output_32_1.png}
    \end{center}

    \hypertarget{la-gente-no-le-da-tanta-importancia-al-estado-del-teluxe9fono}{%
\subparagraph{La gente no le da tanta importancia al estado del
teléfono}\label{la-gente-no-le-da-tanta-importancia-al-estado-del-teluxe9fono}}

    \hypertarget{proporciuxf3n-de-productos-seguxfan-el-estado}{%
\subsubsection{6. Proporción de productos según el
estado}\label{proporciuxf3n-de-productos-seguxfan-el-estado}}

    
    \begin{center}
    \adjustimage{max size={0.9\linewidth}{0.9\paperheight}}{output_35_0.png}
    \end{center}
    
    \hypertarget{hay-una-proporciuxf3n-equilibrada-de-teluxe9fonos-en-buen-muy-buen-y-excelente-estado.-las-tendencias-previamente-analizadas-no-responden-a-una-diferencia-de-oferta}{%
\subparagraph{Hay una proporción equilibrada de teléfonos en buen, muy
buen y excelente estado. Las tendencias previamente analizadas no
responden a una diferencia de
oferta}\label{hay-una-proporciuxf3n-equilibrada-de-teluxe9fonos-en-buen-muy-buen-y-excelente-estado.-las-tendencias-previamente-analizadas-no-responden-a-una-diferencia-de-oferta}}

    \hypertarget{el-estado-del-teluxe9fono-no-es-tan-importante}{%
\subparagraph{El estado del teléfono no es tan
importante}\label{el-estado-del-teluxe9fono-no-es-tan-importante}}

    \hypertarget{a-quuxe9-hora-son-muxe1s-efectivas-las-publicidades}{%
\subsubsection{7. ¿A qué hora son más efectivas las
publicidades?}\label{a-quuxe9-hora-son-muxe1s-efectivas-las-publicidades}}

    \begin{center}
    \adjustimage{max size={0.9\linewidth}{0.9\paperheight}}{output_39_1.png}
    \end{center}
    { \hspace*{\fill} \\}
    
    \hypertarget{los-eventos-que-derivan-en-una-compra}{%
\subsubsection{8. Los eventos que derivan en una
compra}\label{los-eventos-que-derivan-en-una-compra}}

    \hypertarget{para-realizar-este-anuxe1lisis-se-creuxf3-un-nuevo-dataframe-de-sesiones-llamado-flujo_compra.-una-sesiuxf3n-registra-los-movimientos-de-un-usuario-en-un-peruxedodo-de-una-hora.-se-consideruxf3-que-como-muxe1ximo-una-sesiuxf3n-puede-durar-una-hora-es-decir-que-si-un-usuario-entra-a-la-puxe1gina-de-trocafone-en-un-mismo-duxeda-pero-en-distinto-horario-a-la-mauxf1ana-y-a-la-noche-generaruxe1-dos-registros-distintos.-si-bien-solo-estamos-considerando-el-evento-inicial-y-el-final-creemos-que-con-algunas-modificaciones-este-dataframe-puede-muxe1s-adelante-aportar-informaciuxf3n-interesante.}{%
\subparagraph{Para realizar este análisis se creó un nuevo dataframe de
``Sesiones'' llamado ``flujo\_compra''. Una sesión registra los
movimientos de un usuario en un período de una hora. Se consideró que
como máximo una sesión puede durar una hora, es decir que si un usuario
entra a la página de trocafone en un mismo día pero en distinto horario
(a la mañana y a la noche), generará dos registros distintos. Si bien
solo estamos considerando el evento inicial y el final, creemos que con
algunas modificaciones este dataframe puede más adelante aportar
información
interesante.}\label{para-realizar-este-anuxe1lisis-se-creuxf3-un-nuevo-dataframe-de-sesiones-llamado-flujo_compra.-una-sesiuxf3n-registra-los-movimientos-de-un-usuario-en-un-peruxedodo-de-una-hora.-se-consideruxf3-que-como-muxe1ximo-una-sesiuxf3n-puede-durar-una-hora-es-decir-que-si-un-usuario-entra-a-la-puxe1gina-de-trocafone-en-un-mismo-duxeda-pero-en-distinto-horario-a-la-mauxf1ana-y-a-la-noche-generaruxe1-dos-registros-distintos.-si-bien-solo-estamos-considerando-el-evento-inicial-y-el-final-creemos-que-con-algunas-modificaciones-este-dataframe-puede-muxe1s-adelante-aportar-informaciuxf3n-interesante.}}

    
    \begin{center}
    \adjustimage{max size={0.9\linewidth}{0.9\paperheight}}{output_46_1.png}
    \end{center}

    \hypertarget{observamos-que-la-mitad-de-las-ventas-se-llegan-mediante-publicidades}{%
\subparagraph{Observamos que la mitad de las ventas se llegan mediante
publicidades}\label{observamos-que-la-mitad-de-las-ventas-se-llegan-mediante-publicidades}}

    \hypertarget{registraciones-de-los-usuarios}{%
\subsubsection{9. Registraciones de los
usuarios}\label{registraciones-de-los-usuarios}}

   
    \begin{center}
    \adjustimage{max size={0.9\linewidth}{0.9\paperheight}}{output_51_1.png}
    \end{center}
 
    \hypertarget{la-web-incrementuxf3-gradualmente-la-cantidad-de-usuarios-nuevos-por-mes-teniendo-su-pico-en-los-meses-de-junio-y-julio.-tambiuxe9n-se-puede-apreciar-los-duxedas-con-mayor-cantidad-de-registraciones-lunes-martes-y-miuxe9rcoles.}{%
\subparagraph{La web incrementó gradualmente la cantidad de usuarios
nuevos por mes, teniendo su pico en los meses de Junio y Julio. También
se puede apreciar los días con mayor cantidad de registraciones, lunes,
martes y
miércoles.}\label{la-web-incrementuxf3-gradualmente-la-cantidad-de-usuarios-nuevos-por-mes-teniendo-su-pico-en-los-meses-de-junio-y-julio.-tambiuxe9n-se-puede-apreciar-los-duxedas-con-mayor-cantidad-de-registraciones-lunes-martes-y-miuxe9rcoles.}}

    \begin{center}
    \adjustimage{max size={0.9\linewidth}{0.9\paperheight}}{output_53_1.png}
    \end{center}
    { \hspace*{\fill} \\}
    
    \hypertarget{eventos-de-los-no-usuarios}{%
\subsubsection{10. Eventos de los no
usuarios}\label{eventos-de-los-no-usuarios}}


    \begin{Verbatim}[commandchars=\\\{\}]
    
Usuarios con eventos previos al registro:  24022 

Usuarios totales:  26898
    \end{Verbatim}

    \begin{center}
    \adjustimage{max size={0.9\linewidth}{0.9\paperheight}}{output_60_1.png}
    \end{center}
    { \hspace*{\fill} \\}
    
    \hypertarget{las-campauxf1as-de-avisos-online-y-los-buscadores-son-los-principales-eventos-de-los-no-registrados-a-la-web.}{%
\subparagraph{Las campañas de avisos online y los buscadores son los
principales eventos de los no registrados a la
web.}\label{las-campauxf1as-de-avisos-online-y-los-buscadores-son-los-principales-eventos-de-los-no-registrados-a-la-web.}}

   
    \hypertarget{trace-de-los-eventos-de-los-no-registrados-antes-de-ser-hacerse-nuevos-usuarios}{%
\paragraph{Trace de los eventos de los no registrados antes de ser
hacerse nuevos
usuarios}\label{trace-de-los-eventos-de-los-no-registrados-antes-de-ser-hacerse-nuevos-usuarios}}


    \begin{Verbatim}[commandchars=\\\{\}]
Total :  24022
    \end{Verbatim}

\begin{Verbatim}[commandchars=\\\{\}]
                                                    eventos    cantidad   porcentage
         0                                    ad campaign hit      4410   18.358172
         1                                  search engine hit      2590   10.781783
         2                ad campaign hit | search engine hit      1747    7.272500
         3                search engine hit | ad campaign hit      1737    7.230872
         4                   viewed product | ad campaign hit      1299    5.407543
         5                   ad campaign hit | viewed product      1282    5.336775
         6                                    generic listing      1272    5.295146
         7                generic listing | search engine hit      1210    5.037049
         8                search engine hit | generic listing      1071    4.458413
         9  generic listing | search engine hit | ad campaign       530    2.206311
\end{Verbatim}
            
    \hypertarget{observamos-que-aproximadamente-un-18-de-los-usuarios-se-registra-directamente-luego-de-entrar-a-la-puxe1gina-por-una-campauxf1a-de-marketing-y-el-10-luego-de-haber-entrado-por-el-buscador}{%
\subparagraph{Observamos que aproximadamente un \%18 de los usuarios se
registra directamente luego de entrar a la página por una campaña de
marketing y el \%10 luego de haber entrado por el
buscador}\label{observamos-que-aproximadamente-un-18-de-los-usuarios-se-registra-directamente-luego-de-entrar-a-la-puxe1gina-por-una-campauxf1a-de-marketing-y-el-10-luego-de-haber-entrado-por-el-buscador}}

   
    \begin{center}
    \adjustimage{max size={0.9\linewidth}{0.9\paperheight}}{output_67_0.png}
    \end{center}
    { \hspace*{\fill} \\}
    
    \begin{center}
    \adjustimage{max size={0.9\linewidth}{0.9\paperheight}}{output_68_0.png}
    \end{center}
    { \hspace*{\fill} \\}
    
    \hypertarget{se-puede-observar-que-las-campauxf1as-marketing-de-google-y-su-buscador-fue-lo-muxe1s-predominante-para-la-generaciuxf3n-de-usuarios.-tambiuxe9n-se-puede-destacar-criteo-como-campauxf1a-de-marketing.}{%
\subparagraph{Se puede observar que las campañas marketing de Google y
su buscador fue lo más predominante para la generación de usuarios.
También se puede destacar Criteo como campaña de
marketing.}\label{se-puede-observar-que-las-campauxf1as-marketing-de-google-y-su-buscador-fue-lo-muxe1s-predominante-para-la-generaciuxf3n-de-usuarios.-tambiuxe9n-se-puede-destacar-criteo-como-campauxf1a-de-marketing.}}

    \hypertarget{primer-sesiuxf3n-de-los-usuarios-activos-y-casuales}{%
\subsubsection{11. Primer sesión de los usuarios activos y
casuales}\label{primer-sesiuxf3n-de-los-usuarios-activos-y-casuales}}

    \begin{center}
    \adjustimage{max size={0.9\linewidth}{0.9\paperheight}}{output_72_0.png}
    \end{center}
    { \hspace*{\fill} \\}
    
    \hypertarget{podemos-notar-una-similitud-en-los-eventos-de-ambos-tipos-de-usuarios-con-excepciuxf3n-del-checkout-duxf3nde-predominan-los-usuarios-con-solo-una-sesiuxf3n.-esto-no-influye-en-la-compra-de-productos-ya-que-los-usuarios-activos-tienen-mayor-conversiuxf3n-aunque-se-puede-notar-que-en-su-primer-sesiuxf3n-hay-una-muy-pobre-cantidad-de-conversiones.}{%
\subparagraph{Podemos notar una similitud en los eventos de ambos tipos
de usuarios, con excepción del checkout, dónde predominan los usuarios
con solo una sesión. Esto no influye en la compra de productos, ya que
los usuarios `activos' tienen mayor conversión, aunque se puede notar
que en su primer sesión hay una muy pobre cantidad de
conversiones.}\label{podemos-notar-una-similitud-en-los-eventos-de-ambos-tipos-de-usuarios-con-excepciuxf3n-del-checkout-duxf3nde-predominan-los-usuarios-con-solo-una-sesiuxf3n.-esto-no-influye-en-la-compra-de-productos-ya-que-los-usuarios-activos-tienen-mayor-conversiuxf3n-aunque-se-puede-notar-que-en-su-primer-sesiuxf3n-hay-una-muy-pobre-cantidad-de-conversiones.}}

    \hypertarget{tuxe9rminos-muxe1s-utilizados-en-las-buxfasquedas}{%
\subsubsection{12. Términos más utilizados en las
búsquedas}\label{tuxe9rminos-muxe1s-utilizados-en-las-buxfasquedas}}

    \begin{center}
    \adjustimage{max size={0.9\linewidth}{0.9\paperheight}}{output_75_0.png}
    \end{center}
    { \hspace*{\fill} \\}
    
    \hypertarget{observamos-que-lo-muxe1s-buscado-es-la-marca-de-celulares-de-apple-sobre-todo-los-modelos-5s-y-6s.-en-menor-medida-se-busca-marcas-samsung-y-motorola.}{%
\subparagraph{Observamos que lo más buscado es la marca de celulares de
Apple, sobre todo los modelos 5s y 6s. En menor medida se busca marcas
Samsung y
Motorola.}\label{observamos-que-lo-muxe1s-buscado-es-la-marca-de-celulares-de-apple-sobre-todo-los-modelos-5s-y-6s.-en-menor-medida-se-busca-marcas-samsung-y-motorola.}}

    \hypertarget{relacion-entre-los-eventos}{%
\subsubsection{13. Relacion entre los
eventos}\label{relacion-entre-los-eventos}}

    \begin{center}
    \adjustimage{max size={0.9\linewidth}{0.9\paperheight}}{output_78_1.png}
    \end{center}
    { \hspace*{\fill} \\}
    

    % Add a bibliography block to the postdoc
    
    
    
    \end{document}
